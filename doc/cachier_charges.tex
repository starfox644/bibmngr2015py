\documentclass[10pt]{article}
\usepackage[utf8]{inputenc}
\usepackage{graphicx}
\usepackage[top=1cm, bottom=1cm, left=1cm, right=1cm]{geometry}
% \usepackage[T1]{fontenc}

\begin{document}

% \section*{blabla}
\title{Cahier des charges pour bibmanagerpy}
\author{Jérémy}
\maketitle

\section{Description}

Permet de gérer une bibliographie, par ex. articles, livres...

\subsection{Médias gérés}

\begin{itemize}
 \item Articles Scientifiques
 \item Livres
\end{itemize}

\section{Fonctions}

\subsection{Lecture d'un bibtex}

Prend un bibtex en entrée et affiche ses infos.

\subsection{Génération de descriptions}

Génère des descriptions du document dans un certain format de fichier, par ex. latex.
La description comporte les champs principaux comme l'auteur et peut comporter certains champs à remplir comme 
les URL ou la description.

\subsection{Création d'un dossier}

Crée un dossier correspondant à un fichier bibtex fourni en entrée.
Les fichiers de description précédemment décrits sont créés dans ce dossier.

\subsection{Création d'une liste de dossiers}

Crée une liste de dossiers à partir d'une liste de bibtex fournis ou récupérés dans un dossier.

\end{document}




